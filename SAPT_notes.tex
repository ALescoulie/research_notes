\documentclass{article}[letterpaper, margins=1in, 12pt]
\usepackage[a4paper,bindingoffset=0.2in,%
            left=1in,right=1in,top=0.75in,bottom=1in,%
            footskip=.25in]{geometry}
\usepackage{authblk}
\usepackage{amsmath}
\usepackage{graphicx}

\title{Notes on Symmetry Adapted Perturbation Theory}
\author{Alia Lescoulie}

\begin{document}

\maketitle

\section{Fundamentals of SAPT Theory}

Interaction energy $E_{int}$ between two systems $A$ and $B$ typically computed \textit{supermolecular approach} where $E_{int} = E_{AB} - E_{A} -E_{B}$. This approach is very accurate but limited in that is does not break down the factors contributing to the interaction energy. SAPT overcomes this by decomposing the interaction energy into its constituents. Interaction energy is made up of four components:

\begin{itemize}
    \item \textbf{Electrostatics:} The coulomb interactions resulting from charges on the systems.
    \item \textbf{Induction:} The effects of mutual polarization of the molecules.
    \item \textbf{Dispersion:} The effects of fluctuations in electron density creating momentary charges.
    \item \textbf{Exchange:} The short range repulsive forces resulting form the Pauli exclusion principle.
\end{itemize}



\end{document}